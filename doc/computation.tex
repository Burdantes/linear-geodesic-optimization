\documentclass[10pt]{article}

\usepackage{amsfonts, amsmath, amssymb, amsthm}
\usepackage[margin=0.5in]{geometry}
\usepackage{hyperref}

\allowdisplaybreaks
\delimitershortfall-1pt

\newcommand*\delimeter[3]{
	\ensuremath{\mathopen{}\left#2 #1\right#3\mathclose{}{\vphantom{\left#2 #1\right#3}}}
}

\newcommand*\pof[1]{\delimeter{#1}{(}{)}}
\newcommand*\sof[1]{\delimeter{#1}{[}{]}}
\newcommand*\cof[1]{\delimeter{#1}{\{}{\}}}
\newcommand*\aof[1]{\delimeter{#1}{\langle}{\rangle}}

\newcommand*\abs[1]{\delimeter{#1}{|}{|}}
\newcommand*\norm[1]{\delimeter{#1}{\|}{\|}}
\newcommand*\floor[1]{\delimeter{#1}{\lfloor}{\rfloor}}
\newcommand*\ceil[1]{\delimeter{#1}{\lceil}{\rceil}}

\newcommand*\ooint[1]{\delimeter{#1}{(}{)}}
\newcommand*\ocint[1]{\delimeter{#1}{(}{]}}
\newcommand*\coint[1]{\delimeter{#1}{[}{)}}
\newcommand*\ccint[1]{\delimeter{#1}{[}{]}}

\newcommand*\eval[1]{\delimeter{#1}{.}{|}}

\newcommand*\Dif{\ensuremath{\mathrm{D}}}
\newcommand*\dif{\ensuremath{\mathrm{d}}}

\begin{document}
	\pagenumbering{gobble}

	\section{Problem Setup}

	As input, we are given a graph \(G = \pof{V_G, E_G}\), where each vertex is a geographic position \(s_i \in S^2\), and each edge \(\pof{i, j}\) has an associated (Olivier-Ricci) curvature \(R_{i, j} \in \ooint{-2, 1}\) and an associated latency \(t_{i, j} \in \mathbb{R}_{\ge 0}\).

	Intuitively, we want to return a surface in \(\mathbb{R}^3\) that is the graph of a function \(\rho : S^2 \to \mathbb{R}_{> 0}\) whose geodesics \(g_{i, j}\) between \(s_i\) and \(s_j\) (and their missing \(\rho\)-coordinates) have length \(\phi_{i, j}\) that is in a linear relationship with the latency.

	The strategy to realize this intuition is to create a mesh \(M = \pof{V_M, E_M}\) supported on a subset of \(S^2\) that contains our input positions. Let \(P\) be the support. Then for each \(s_i \in P\), we want to assign a \(\rho_i \in \mathbb{R}_{> 0}\), which in turn gives a point \(v_i = \pof{s_i, \rho_i} \in V\). This setup is made explicit in \texttt{mesh/sphere.py}. A similar setup is found in \texttt{mesh/rectangle.py}, where we use \(\ccint{0, 1}^2\) instead of \(S^2\).

	\section{Objective/Loss Functions}

	To enforce that the mesh approximates our desired surface, we define the objective functions \begin{align*}
		\mathcal{L}_{\mathrm{geodesic}}\pof{M} &\triangleq \sum_{e \in E_G} \pof{\text{least squares residual of edge \(e\)}}^2, \\
		\mathcal{L}_{\mathrm{smooth}}\pof{M} &\triangleq -\rho^\intercal L_C\rho, \\
		\mathcal{L}_{\mathrm{curvature}}\pof{M} &\triangleq \sum_{\substack{v \in V_M \\ \text{\(v\) close to \(\pof{i, j}\)}}} \pof{\kappa\pof{v} - R_{i, j}}^2, \\
		\mathcal{L}\pof{M} &\triangleq \lambda_{\mathrm{geodesic}}\mathcal{L}_{\mathrm{geodesic}}\pof{M} + \lambda_{\mathrm{curvature}}\mathcal{L}_{\mathrm{curvature}}\pof{M} + \lambda_{\mathrm{smooth}}\mathcal{L}_{\mathrm{smooth}}\pof{M},
	\end{align*} where the \(\lambda\)'s are tunable hyperparameters. The other variables will be defined in the upcoming subsections. Our goal is then to minimize \(\mathcal{L}\pof{M}\).

	Note that the loss functions (particularly the geodesic and total ones) also have a dependence on the measured latencies. We omit that as a written parameter because they are treated as fixed (we are really optimizing over the manifold, not over the measured latencies).

	\subsection{Laplacian}
	Some mesh notation first. If \(i\) and \(j\) are two indices vertices for which \(\pof{v_i, v_j} \in E_M\), let \(c\pof{i, j}\) be the index such that \(v_i \to v_j \to v_{c\pof{i, j}}\) traces a triangle counterclockwise. Note that this index exists and is unique assuming we have a mesh without boundary. On a mesh with boundary, if no \(c\pof{i, j}\) exists, then the half-edge \(\pof{v_i, v_j}\) lies on the boundary.

We also write \(\partial M\) to represent the boundary of our mesh. Abusing notation, we can write things like \(v_i \in \partial M\) or \(\pof{v_i, v_j} \in \partial M\).

We define the following variables: \begin{center}\begin{tabular}{r|l}
	\(N_{i, j}\) & Outward normal of triangle \(v_i \to v_j \to v_{c\pof{i, j}}\) \\ \hline
	\(A_{i, j}\) & Area of triangle \(v_i \to v_j \to v_{c\pof{i, j}}\) \\ \hline
	\(D_{i, j}\) & Vertex triangle areas; diagonal \\ \hline
	\(\theta_{i, j}\) & Measure of \(\angle v_iv_{c\pof{i, j}}v_j\) \\ \hline
	\(L_C^{\text{N}}\) & Cotangent operator with \href{https://en.wikipedia.org/wiki/Neumann_boundary_condition}{zero-Neumann boundary condition} \\ \hline
	\(L_C^{\text{D}}\) & Cotangent operator with \href{https://en.wikipedia.org/wiki/Dirichlet_boundary_condition}{zero-Dirichlet boundary condition} \\ \hline
	\(L_C\) & Cotangent operator in the no-boundary case; sparse
\end{tabular}\end{center}

\subsubsection{Forward Computation}

We have the following (standard) definition of the Laplace-Beltrami operator on a mesh:

\begin{align*}
	N_{i, j} &= \pof{v_i - v_{c\pof{i, j}}} \times \pof{v_j - v_{c\pof{i, j}}}, \\
	A_{i, j} &= \frac{1}{2}\norm{N_{i, j}}_2, \\
	D_{i, j} &= \begin{cases}
		\frac{1}{3}{\displaystyle\sum_{\substack{k \\ \pof{v_i, v_k} \in E_M}}A_{i, k}} & \text{if \(i = j\)}, \\
		0 & \text{otherwise},
	\end{cases} \\
	\cot\pof{\theta_{i, j}} &= \frac{\pof{v_i - v_{c\pof{i, j}}} \cdot \pof{v_j - v_{c\pof{i, j}}}}{2A_{i, j}}, \\
	\pof{L_C^{\text{N}}}_{i, j} &= \begin{cases}
		\frac{1}{2}\cot\pof{\theta_{i, j}} & \text{if \(\pof{v_i, v_j} \in \partial M\)}, \\
		\frac{1}{2}\cot\pof{\theta_{j, i}} & \text{if \(\pof{v_j, v_i} \in \partial M\)}, \\
		\frac{1}{2}\pof{\cot\pof{\theta_{i, j}} + \cot\pof{\theta_{j, i}}} & \text{if \(\pof{v_i, v_j} \in E_M\) and \(\pof{v_j, v_i} \in E_M\)}, \\
		-\frac{1}{2}\pof{\displaystyle\sum_{\substack{k \\ \pof{v_i, v_k} \in E_M}}\cot\pof{\theta_{i, k}} + \sum_{\substack{k \\ \pof{v_k, v_i} \in E_M}}\cot\pof{\theta_{k, i}}} & \text{if \(i = j\)}, \\
		0 & \text{otherwise},
	\end{cases} \\
	\pof{L_C^{\text{D}}}_{i, j} &= \begin{cases}
		\frac{1}{2}\pof{\cot\pof{\theta_{i, j}} + \cot\pof{\theta_{j, i}}} & \text{if \(\pof{v_i, v_j} \in E_M\), \(v_i \not\in \partial M\), and \(v_j \not\in \partial M\)}, \\
		-\frac{1}{2}{\displaystyle\sum_{\substack{k \not\in \partial M \\ \pof{v_i, v_k} \in E_M \\ \pof{v_k, v_i} \in E_M}}\pof{\cot\pof{\theta_{i, k}} + \cot\pof{\theta_{k, i}}}} & \text{if \(i = j\) and \(v_i \not\in \partial M\)}, \\
		0 & \text{otherwise}.
	\end{cases}
\end{align*}

Flipping our attention back to meshes without boundary, the two definitions above coincide, so we can write \[L_C = L_C^{\text{Neumann}} = L_C^{\text{Dirichlet}}.\]

\subsubsection{Reverse Computation}

For the ease of notation, assume that we are using the spherical setup, so \(v_\ell = \rho_\ell s_\ell\).

We compute

\begin{align*}
	\frac{\partial v_i}{\partial \rho_\ell} &= \begin{cases}
		s_i & \text{if \(\ell = i\)}, \\
		0 & \text{otherwise},
	\end{cases} \\
	\frac{\partial N_{i, j}}{\partial \rho_\ell} &= \begin{cases}
		\pof{v_{c\pof{i, j}} - v_j} \times \frac{\partial v_\ell}{\partial \rho_\ell} & \text{if \(\ell = i\)}, \\
		\pof{v_i - v_{c\pof{i, j}}} \times \frac{\partial v_\ell}{\partial \rho_\ell} & \text{if \(\ell = j\)}, \\
		\pof{v_j - v_i} \times \frac{\partial v_\ell}{\partial \rho_\ell} & \text{if \(\ell = c\pof{i, j}\)}, \\
		0 & \text{otherwise},
	\end{cases} \\
	\frac{\partial A_{i, j}}{\partial \rho_\ell} &= \frac{1}{4A_{i, j}}N_{i, j} \cdot \frac{\partial N_{i, j}}{\partial \rho_\ell}, \\
	\pof{\frac{\partial D}{\partial \rho_\ell}}_{i, j} &= \begin{cases}
		\frac{1}{3}{\displaystyle\sum_{\substack{k \\ \pof{v_i, v_k} \in E_M}}\frac{\partial A_{i, k}}{\partial \rho_\ell}} & \text{if \(i = j\)}, \\
		0 & \text{otherwise},
	\end{cases} \\
	\frac{\partial}{\partial \rho_\ell}\cot\pof{\theta_{i, j}} &= \begin{cases}
		\displaystyle\frac{\pof{v_j - v_{c\pof{i, j}}} \cdot \frac{\partial v_\ell}{\partial \rho_\ell} - 2\cot\pof{\theta_{i, j}}\frac{\partial A_{i, j}}{\partial \rho_\ell}}{2A_{i, j}} & \text{if \(\ell = i\)}, \\
		\displaystyle\frac{\pof{v_i - v_{c\pof{i, j}}} \cdot \frac{\partial v_\ell}{\partial \rho_\ell} - 2\cot\pof{\theta_{i, j}}\frac{\partial A_{i, j}}{\partial \rho_\ell}}{2A_{i, j}} & \text{if \(\ell = j\)}, \\
		\displaystyle\frac{\pof{2v_{c\pof{i, j}} - v_i - v_j} \cdot \frac{\partial v_\ell}{\partial \rho_\ell} - 2\cot\pof{\theta_{i, j}}\frac{\partial A_{i, j}}{\partial \rho_\ell}}{2A_{i, j}} & \text{if \(\ell = c\pof{i, j}\)}, \\
		0 & \text{otherwise},
	\end{cases} \\
	\pof{\frac{\partial L_C^{\text{N}}}{\partial \rho_\ell}}_{i, j} &= \begin{cases}
		\frac{1}{2}\frac{\partial}{\partial \rho_\ell}\cot\pof{\theta_{i, j}} & \text{if \(\pof{v_i, v_j} \in \partial M\)}, \\
		\frac{1}{2}\frac{\partial}{\partial \rho_\ell}\cot\pof{\theta_{j, i}} & \text{if \(\pof{v_j, v_i} \in \partial M\)}, \\
		\frac{1}{2}\pof{\frac{\partial}{\partial \rho_\ell}\cot\pof{\theta_{i, j}} + \frac{\partial}{\partial \rho_\ell}\cot\pof{\theta_{j, i}}} & \text{if \(\pof{v_i, v_j} \in E_M\) and \(\pof{v_j, v_i} \in E_M\)}, \\
		-\frac{1}{2}\pof{\displaystyle\sum_{\substack{k \\ \pof{v_i, v_k} \in E_M}}\frac{\partial}{\partial \rho_\ell}\cot\pof{\theta_{i, k}} + \sum_{\substack{k \\ \pof{v_k, v_i} \in E_M}}\frac{\partial}{\partial \rho_\ell}\cot\pof{\theta_{k, i}}} & \text{if \(i = j\)}, \\
		0 & \text{otherwise},
	\end{cases} \\
	\pof{\frac{\partial L_C^{\text{D}}}{\partial \rho_\ell}}_{i, j} &= \begin{cases}
		\frac{1}{2}\pof{\frac{\partial}{\partial \rho_\ell}\cot\pof{\theta_{i, j}} + \frac{\partial}{\partial \rho_\ell}\cot\pof{\theta_{j, i}}} & \text{if \(\pof{v_i, v_j} \in E_M\), \(v_i \not\in \partial M\), and \(v_j \not\in \partial M\)}, \\
		-\frac{1}{2}{\displaystyle\sum_{\substack{k \not\in \partial M \\ \pof{v_i, v_k} \in E_M \\ \pof{v_k, v_i} \in E_M}}\pof{\frac{\partial}{\partial \rho_\ell}\cot\pof{\theta_{i, k}} + \frac{\partial}{\partial \rho_\ell}\cot\pof{\theta_{k, i}}}} & \text{if \(i = j\) and \(v_i \not\in \partial M\)}, \\
		0 & \text{otherwise}.
	\end{cases}
\end{align*}

\end{document}
