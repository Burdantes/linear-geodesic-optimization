\documentclass[10pt]{article}

\usepackage{amsfonts, amsmath, amssymb, amsthm}
\usepackage[margin=0.5in]{geometry}
\usepackage{hyperref}
\usepackage{xcolor}

\hypersetup{
    colorlinks=true,
    linkcolor=blue,
    filecolor=magenta,
    urlcolor=cyan,
    pdftitle={Computation Documentation}
}

\allowdisplaybreaks
\delimitershortfall-1pt

\newcommand*\delimeter[3]{
	\ensuremath{\mathopen{}\left#2 #1\right#3\mathclose{}{\vphantom{\left#2 #1\right#3}}}
}

\newcommand*\pof[1]{\delimeter{#1}{(}{)}}
\newcommand*\sof[1]{\delimeter{#1}{[}{]}}
\newcommand*\cof[1]{\delimeter{#1}{\{}{\}}}
\newcommand*\aof[1]{\delimeter{#1}{\langle}{\rangle}}

\newcommand*\abs[1]{\delimeter{#1}{|}{|}}
\newcommand*\norm[1]{\delimeter{#1}{\|}{\|}}
\newcommand*\floor[1]{\delimeter{#1}{\lfloor}{\rfloor}}
\newcommand*\ceil[1]{\delimeter{#1}{\lceil}{\rceil}}

\newcommand*\ooint[1]{\delimeter{#1}{(}{)}}
\newcommand*\ocint[1]{\delimeter{#1}{(}{]}}
\newcommand*\coint[1]{\delimeter{#1}{[}{)}}
\newcommand*\ccint[1]{\delimeter{#1}{[}{]}}

\newcommand*\eval[1]{\delimeter{#1}{.}{|}}

\newcommand*\Dif{\ensuremath{\mathrm{D}}}
\newcommand*\dif{\ensuremath{\mathrm{d}}}

\begin{document}
	\pagestyle{empty}

	\section{Problem Setup}

	As input, we are given a directed graph \(G = \pof{V_G, E_G}\), where each vertex is a geographic position \(s_i \in S^2\), and each edge \(\pof{i, j}\) has an associated (Olivier-Ricci) curvature \(R_{i, j} \in \ooint{-2, 1}\) and an associated latency \(t_{i, j} \in \mathbb{R}_{\ge 0}\).

	Intuitively, we want to return a surface in \(\mathbb{R}^3\) that is the graph of a function \(\rho : S^2 \to \mathbb{R}_{> 0}\) whose geodesics \(g_{i, j}\) between \(s_i\) and \(s_j\) (and their missing \(\rho\)-coordinates) have length \(\phi_{i, j}\) that is in a linear relationship with the latency.

	The strategy to realize this intuition is to create a mesh \(M = \pof{V_M, E_M}\) supported on a subset of \(S^2\) that contains our input positions \(V_G\). We use a standard \href{https://en.wikipedia.org/wiki/Doubly_connected_edge_list}{half-edge} setup, so that \(E_M\) is a set of ordered pairs (edges are directed). Let \(P\) be the support. Then for each \(s_i \in P\), we want to assign a \(\rho_i \in \mathbb{R}_{> 0}\), which in turn gives a point \(v_i = \pof{s_i, \rho_i} \in V\). This setup is made explicit in \texttt{mesh/sphere.py}.

	A similar setup is found in \texttt{mesh/rectangle.py}, where we use \(\ccint{0, 1}^2\) instead of \(S^2\). In general, this setup just requires that the position of any mesh vertex is controlled by a single scalar parameter.

	\section{Objective/Loss Functions}

	To enforce that the mesh approximates our desired surface, we define the objective functions \begin{align*}
		\mathcal{L}_{\mathrm{geodesic}}\pof{M} &\triangleq \sum_{e \in E_G} \pof{\text{least squares residual of edge \(e\)}}^2, \\
		\mathcal{L}_{\mathrm{smooth}}\pof{M} &\triangleq -\rho^\intercal L_C\rho, \\
		\mathcal{L}_{\mathrm{curvature}}\pof{M} &\triangleq \sum_{\substack{v \in V_M \\ \text{\(v\) close to \(\pof{i, j}\)}}} \pof{\kappa\pof{v} - R_{i, j}}^2, \\
		\mathcal{L}\pof{M} &\triangleq \lambda_{\mathrm{geodesic}}\mathcal{L}_{\mathrm{geodesic}}\pof{M} + \lambda_{\mathrm{curvature}}\mathcal{L}_{\mathrm{curvature}}\pof{M} + \lambda_{\mathrm{smooth}}\mathcal{L}_{\mathrm{smooth}}\pof{M},
	\end{align*} where the \(\lambda\)'s are tunable hyperparameters. The other variables will be defined in the upcoming subsections. Our goal is then to minimize \(\mathcal{L}\pof{M}\).

	Note that the loss functions (particularly the geodesic and total ones) also have a dependence on the measured latencies. We omit that as a written parameter because they are treated as fixed (we are really optimizing over the manifold, not over the measured latencies).

	\subsection{Laplacian}
	Some mesh notation first. If \(i\) and \(j\) are two indices vertices for which \(\pof{v_i, v_j} \in E_M\), let \(c\pof{i, j}\) be the index such that \(v_i \to v_j \to v_{c\pof{i, j}}\) traces a triangle counterclockwise. Note that this index exists and is unique assuming we have a mesh without boundary. On a mesh with boundary, if no \(c\pof{i, j}\) exists, then the half-edge \(\pof{v_i, v_j}\) lies on the boundary.

We also write \(\partial M\) to represent the boundary of our mesh. Abusing notation, we can write things like \(v_i \in \partial M\) or \(\pof{v_i, v_j} \in \partial M\).

We define the following variables: \begin{center}\begin{tabular}{r|l}
	\(N_{i, j}\) & Outward normal of triangle \(v_i \to v_j \to v_{c\pof{i, j}}\) \\ \hline
	\(A_{i, j}\) & Area of triangle \(v_i \to v_j \to v_{c\pof{i, j}}\) \\ \hline
	\(D_{i, j}\) & Vertex triangle areas; diagonal \\ \hline
	\(\theta_{i, j}\) & Measure of \(\angle v_iv_{c\pof{i, j}}v_j\) \\ \hline
	\(L_C^{\text{N}}\) & Cotangent operator with \href{https://en.wikipedia.org/wiki/Neumann_boundary_condition}{zero-Neumann boundary condition} \\ \hline
	\(L_C^{\text{D}}\) & Cotangent operator with \href{https://en.wikipedia.org/wiki/Dirichlet_boundary_condition}{zero-Dirichlet boundary condition} \\ \hline
	\(L_C\) & Cotangent operator in the no-boundary case; sparse
\end{tabular}\end{center}

\subsubsection{Forward Computation}

We have the following (standard) definition of the Laplace-Beltrami operator on a mesh:

\begin{align*}
	N_{i, j} &= \pof{v_i - v_{c\pof{i, j}}} \times \pof{v_j - v_{c\pof{i, j}}}, \\
	A_{i, j} &= \frac{1}{2}\norm{N_{i, j}}_2, \\
	D_{i, j} &= \begin{cases}
		\frac{1}{3}{\displaystyle\sum_{\substack{k \\ \pof{v_i, v_k} \in E_M}}A_{i, k}} & \text{if \(i = j\)}, \\
		0 & \text{otherwise},
	\end{cases} \\
	\cot\pof{\theta_{i, j}} &= \frac{\pof{v_i - v_{c\pof{i, j}}} \cdot \pof{v_j - v_{c\pof{i, j}}}}{2A_{i, j}}, \\
	\pof{L_C^{\text{N}}}_{i, j} &= \begin{cases}
		\frac{1}{2}\cot\pof{\theta_{i, j}} & \text{if \(\pof{v_i, v_j} \in \partial M\)}, \\
		\frac{1}{2}\cot\pof{\theta_{j, i}} & \text{if \(\pof{v_j, v_i} \in \partial M\)}, \\
		\frac{1}{2}\pof{\cot\pof{\theta_{i, j}} + \cot\pof{\theta_{j, i}}} & \text{if \(\pof{v_i, v_j} \in E_M\) and \(\pof{v_j, v_i} \in E_M\)}, \\
		-\frac{1}{2}\pof{\displaystyle\sum_{\substack{k \\ \pof{v_i, v_k} \in E_M}}\cot\pof{\theta_{i, k}} + \sum_{\substack{k \\ \pof{v_k, v_i} \in E_M}}\cot\pof{\theta_{k, i}}} & \text{if \(i = j\)}, \\
		0 & \text{otherwise},
	\end{cases} \\
	\pof{L_C^{\text{D}}}_{i, j} &= \begin{cases}
		\frac{1}{2}\pof{\cot\pof{\theta_{i, j}} + \cot\pof{\theta_{j, i}}} & \text{if \(\pof{v_i, v_j} \in E_M\), \(v_i \not\in \partial M\), and \(v_j \not\in \partial M\)}, \\
		-\frac{1}{2}{\displaystyle\sum_{\substack{k \not\in \partial M \\ \pof{v_i, v_k} \in E_M \\ \pof{v_k, v_i} \in E_M}}\pof{\cot\pof{\theta_{i, k}} + \cot\pof{\theta_{k, i}}}} & \text{if \(i = j\) and \(v_i \not\in \partial M\)}, \\
		0 & \text{otherwise}.
	\end{cases}
\end{align*}

Flipping our attention back to meshes without boundary, the two definitions above coincide, so we can write \[L_C = L_C^{\text{Neumann}} = L_C^{\text{Dirichlet}}.\]

\subsubsection{Reverse Computation}

For the ease of notation, assume that we are using the spherical setup, so \(v_\ell = \rho_\ell s_\ell\).

We compute

\begin{align*}
	\frac{\partial v_i}{\partial \rho_\ell} &= \begin{cases}
		s_i & \text{if \(\ell = i\)}, \\
		0 & \text{otherwise},
	\end{cases} \\
	\frac{\partial N_{i, j}}{\partial \rho_\ell} &= \begin{cases}
		\pof{v_{c\pof{i, j}} - v_j} \times \frac{\partial v_\ell}{\partial \rho_\ell} & \text{if \(\ell = i\)}, \\
		\pof{v_i - v_{c\pof{i, j}}} \times \frac{\partial v_\ell}{\partial \rho_\ell} & \text{if \(\ell = j\)}, \\
		\pof{v_j - v_i} \times \frac{\partial v_\ell}{\partial \rho_\ell} & \text{if \(\ell = c\pof{i, j}\)}, \\
		0 & \text{otherwise},
	\end{cases} \\
	\frac{\partial A_{i, j}}{\partial \rho_\ell} &= \frac{1}{4A_{i, j}}N_{i, j} \cdot \frac{\partial N_{i, j}}{\partial \rho_\ell}, \\
	\pof{\frac{\partial D}{\partial \rho_\ell}}_{i, j} &= \begin{cases}
		\frac{1}{3}{\displaystyle\sum_{\substack{k \\ \pof{v_i, v_k} \in E_M}}\frac{\partial A_{i, k}}{\partial \rho_\ell}} & \text{if \(i = j\)}, \\
		0 & \text{otherwise},
	\end{cases} \\
	\frac{\partial}{\partial \rho_\ell}\cot\pof{\theta_{i, j}} &= \begin{cases}
		\displaystyle\frac{\pof{v_j - v_{c\pof{i, j}}} \cdot \frac{\partial v_\ell}{\partial \rho_\ell} - 2\cot\pof{\theta_{i, j}}\frac{\partial A_{i, j}}{\partial \rho_\ell}}{2A_{i, j}} & \text{if \(\ell = i\)}, \\
		\displaystyle\frac{\pof{v_i - v_{c\pof{i, j}}} \cdot \frac{\partial v_\ell}{\partial \rho_\ell} - 2\cot\pof{\theta_{i, j}}\frac{\partial A_{i, j}}{\partial \rho_\ell}}{2A_{i, j}} & \text{if \(\ell = j\)}, \\
		\displaystyle\frac{\pof{2v_{c\pof{i, j}} - v_i - v_j} \cdot \frac{\partial v_\ell}{\partial \rho_\ell} - 2\cot\pof{\theta_{i, j}}\frac{\partial A_{i, j}}{\partial \rho_\ell}}{2A_{i, j}} & \text{if \(\ell = c\pof{i, j}\)}, \\
		0 & \text{otherwise},
	\end{cases} \\
	\pof{\frac{\partial L_C^{\text{N}}}{\partial \rho_\ell}}_{i, j} &= \begin{cases}
		\frac{1}{2}\frac{\partial}{\partial \rho_\ell}\cot\pof{\theta_{i, j}} & \text{if \(\pof{v_i, v_j} \in \partial M\)}, \\
		\frac{1}{2}\frac{\partial}{\partial \rho_\ell}\cot\pof{\theta_{j, i}} & \text{if \(\pof{v_j, v_i} \in \partial M\)}, \\
		\frac{1}{2}\pof{\frac{\partial}{\partial \rho_\ell}\cot\pof{\theta_{i, j}} + \frac{\partial}{\partial \rho_\ell}\cot\pof{\theta_{j, i}}} & \text{if \(\pof{v_i, v_j} \in E_M\) and \(\pof{v_j, v_i} \in E_M\)}, \\
		-\frac{1}{2}\pof{\displaystyle\sum_{\substack{k \\ \pof{v_i, v_k} \in E_M}}\frac{\partial}{\partial \rho_\ell}\cot\pof{\theta_{i, k}} + \sum_{\substack{k \\ \pof{v_k, v_i} \in E_M}}\frac{\partial}{\partial \rho_\ell}\cot\pof{\theta_{k, i}}} & \text{if \(i = j\)}, \\
		0 & \text{otherwise},
	\end{cases} \\
	\pof{\frac{\partial L_C^{\text{D}}}{\partial \rho_\ell}}_{i, j} &= \begin{cases}
		\frac{1}{2}\pof{\frac{\partial}{\partial \rho_\ell}\cot\pof{\theta_{i, j}} + \frac{\partial}{\partial \rho_\ell}\cot\pof{\theta_{j, i}}} & \text{if \(\pof{v_i, v_j} \in E_M\), \(v_i \not\in \partial M\), and \(v_j \not\in \partial M\)}, \\
		-\frac{1}{2}{\displaystyle\sum_{\substack{k \not\in \partial M \\ \pof{v_i, v_k} \in E_M \\ \pof{v_k, v_i} \in E_M}}\pof{\frac{\partial}{\partial \rho_\ell}\cot\pof{\theta_{i, k}} + \frac{\partial}{\partial \rho_\ell}\cot\pof{\theta_{k, i}}}} & \text{if \(i = j\) and \(v_i \not\in \partial M\)}, \\
		0 & \text{otherwise}.
	\end{cases}
\end{align*}


	\subsection{Geodesic Distance via the Heat Method}
	Here are the variables used for this part of the computation: \begin{center}\begin{tabular}{r|l}
	\(\gamma\) & Set of points in \(V_M\) \\ \hline
	\(h\) & Mean half-edge length \\ \hline
	\(\delta^\gamma\) & Heat source (indicator on \(\gamma\)) \\ \hline
	\(u^{\gamma, \text{N}}\) & Heat flow with zero-Neumann boundary condition \\ \hline
	\(u^{\gamma, \text{D}}\) & Heat flow with zero-Dirichlet boundary condition \\ \hline
	\(u^\gamma\) & Heat flow \\ \hline
	\(q^\gamma_{i, j}\) & Intermediate value for computation \\ \hline
	\(m^\gamma_{i, j}\) & Intermediate value for computation \\ \hline
	\(X^\gamma_{i, j}\) & Unit vector in same direction as \(\nabla u^\gamma_{i, j}\) \\ \hline
	\(p^\gamma_{i, j}\) & Intermediate value for computation \\ \hline
	\(\widetilde{\phi}^\gamma\) & Vector of offset geodesic distances \\ \hline
	\(\phi^\gamma\) & Vector of offset geodesic distances
\end{tabular}\end{center}

\subsubsection{Forward Computation}

Say we want to find the geodesic distances to a set of points \(\gamma \subseteq V_M\). Following the \href{https://www.cs.cmu.edu/~kmcrane/Projects/HeatMethod/}{Crane et al's Heat Method}, we use the (approximate) heat flow \(u^\gamma\), where

\begin{align*}
	h &= \frac{1}{\abs{E_M}}\sum_{\substack{i, j \\ \pof{v_i, v_j} \in E_M}}\norm{v_i - v_j}_2, \\
	\delta^\gamma &= \begin{cases}
		1 & \text{if \(v_i \in \gamma\)}, \\
		0 & \text{if \(v_i \not\in \gamma\)},
	\end{cases} \\
	u^{\gamma, \text{N}} &= \pof{D - h^2L_C^{\text{N}}}^{-1}\delta^\gamma, \\
	u^{\gamma, \text{D}} &= \pof{D - h^2L_C^{\text{D}}}^{-1}\delta^\gamma, \\
	u^\gamma &= \frac{1}{2}\pof{u^{\gamma, \text{N}} + u^{\gamma, \text{D}}}, \\
	q^\gamma_{i, j} &= u^\gamma_i\pof{v_{c\pof{i, j}} - v_j}, \\
	m^\gamma_{i, j} &= q^\gamma_{i, j} + q^\gamma_{j, c\pof{i, j}} + q^\gamma_{c\pof{i, j}, i}, \\
	\pof{\nabla u^\gamma}_{i, j} &= N_{i, j} \times m^\gamma_{i, j}, \\
	X^\gamma_{i, j} &= -\frac{\pof{\nabla u^\gamma}_{i, j}}{\norm{\pof{\nabla u^\gamma}_{i, j}}_2}, \\
	p_{i, j} &= \cot\pof{\theta_{i, j}}\pof{v_j - v_i}, \\
	\pof{\nabla \cdot X^\gamma}_i &= \frac{1}{2}\sum_{\substack{k \\ \pof{v_i, v_k} \in E_M}}\pof{p_{i, k} - p_{c\pof{i, k}, i}} \cdot X^\gamma_{i, k}, \\
	\widetilde{\phi}^\gamma &= \pof{L_C^{\text{N}}}^+ \cdot \pof{\nabla \cdot X^\gamma}, \\
	\phi^\gamma &= \widetilde{\phi}^\gamma - \min\pof{\widetilde{\phi}^\gamma}.
\end{align*}

Here, \(\pof{L_C^{\text{N}}}^+\) is the \href{https://en.wikipedia.org/wiki/Moore%E2%80%93Penrose_inverse}{pseudoinverse} of \(L_C^{\text{N}}\) (this is necessary as it is singular).

Note that we're being careful about which pieces have a dependence on \(\gamma\), as we can reuse certain computations if we want to compute distances from multiple sources. We can get the pairwise distance matrix (that is, get rid of the \(\gamma\) dependence) from \[\phi_{i, j} = \pof{\phi^{\cof{v_j}}}_i.\]

\subsubsection{Reverse Computation}

Note that \(c\pof{i, c\pof{j, i}} = j\). This is helpful for reindexing some sums (in particular, the one for \(\nabla \cdot X\)).

We then have the following partial derivatives:

\begin{align*}
	\frac{\partial h}{\partial \rho_\ell} &= \frac{1}{\abs{E_M}}\pof{\sum_{\substack{k \\ \pof{v_\ell, v_k} \in E_M}} \frac{\pof{v_\ell - v_k}}{\norm{v_\ell - v_k}_2} \cdot \frac{\partial v_\ell}{\partial \rho_\ell} + \sum_{\substack{k \\ \pof{v_k, v_\ell} \in E_M}} \frac{\pof{v_\ell - v_k}}{\norm{v_\ell - v_k}_2} \cdot \frac{\partial v_\ell}{\partial \rho_\ell}}, \\
	\frac{\partial u^{\gamma, \text{N}}}{\partial \rho_\ell} &= -\pof{D - h^2L_C^{\text{N}}}^{-1}\pof{\frac{\partial D}{\partial \rho_\ell} - 2h\frac{\partial h}{\partial \rho_\ell}L_C^{\text{N}} - h^2\frac{\partial L_C^{\text{N}}}{\partial \rho_\ell}}u^{\gamma, \text{N}}, \\
	\frac{\partial u^{\gamma, \text{D}}}{\partial \rho_\ell} &= -\pof{D - h^2L_C^{\text{D}}}^{-1}\pof{\frac{\partial D}{\partial \rho_\ell} - 2h\frac{\partial h}{\partial \rho_\ell}L_C^{\text{D}} - h^2\frac{\partial L_C^{\text{D}}}{\partial \rho_\ell}}u^{\gamma, \text{D}}, \\
	\frac{\partial u^\gamma}{\partial \rho_\ell} &= \frac{1}{2}\pof{\frac{\partial u^{\gamma, \text{N}}}{\partial \rho_\ell} + \frac{\partial u^{\gamma, \text{D}}}{\partial \rho_\ell}}, \\
	\frac{\partial q^\gamma_{i, j}}{\partial \rho_\ell} &= \begin{cases}
		\frac{\partial u^\gamma_i}{\rho_\ell}\pof{v_{c\pof{i, j}} - v_j} - u^\gamma_i\frac{\partial v_\ell}{\rho_\ell} & \text{if \(\ell = j\)}, \\
		\frac{\partial u^\gamma_i}{\rho_\ell}\pof{v_{c\pof{i, j}} - v_j} + u^\gamma_i\frac{\partial v_\ell}{\partial \rho_\ell} & \text{if \(\ell = c\pof{i, j}\)}, \\
		\frac{\partial u^\gamma_i}{\rho_\ell}\pof{v_{c\pof{i, j}} - v_j} & \text{otherwise},
	\end{cases} \\
	\frac{\partial m^\gamma_{i, j}}{\partial \rho_\ell} &= \frac{\partial q^\gamma_{i, j}}{\partial \rho_\ell} + \frac{\partial q^\gamma_{j, c\pof{i, j}}}{\partial \rho_\ell} + \frac{\partial q^\gamma_{c\pof{i, j}, i}}{\partial \rho_\ell}, \\
	\frac{\partial \pof{\nabla u^\gamma}_{i, j}}{\partial \rho_\ell} &= \frac{\partial N_{i, j}}{\partial \rho_\ell} \times m^\gamma_{i, j} + N_{i, j} \times \frac{\partial m^\gamma_{i, j}}{\partial \rho_\ell}, \\
	\frac{\partial X^\gamma_{i, j}}{\partial \rho_\ell} &= -\frac{1}{\norm{\pof{\nabla u^\gamma}_{i, j}}_2}\pof{I - X^\gamma_{i, j}\pof{X^\gamma_{i, j}}^\intercal}\frac{\partial \pof{\nabla u^\gamma}_{i, j}}{\partial \rho_\ell}, \\
	\frac{\partial p_{i, j}}{\partial \rho} &= \begin{cases}
		\pof{\frac{\partial}{\partial \rho_\ell}\cot\pof{\theta_{i, j}}}\pof{v_j - v_i} - \cot\pof{\theta_{i, j}}\frac{\partial v_\ell}{\rho_\ell} & \text{if \(\ell = i\)}, \\
		\pof{\frac{\partial}{\partial \rho_\ell}\cot\pof{\theta_{i, j}}}\pof{v_j - v_i} + \cot\pof{\theta_{i, j}}\frac{\partial v_\ell}{\rho_\ell} & \text{if \(\ell = j\)}, \\
		\pof{\frac{\partial}{\partial \rho_\ell}\cot\pof{\theta_{i, j}}}\pof{v_j - v_i} & \text{if \(\ell = c\pof{i, j}\)}, \\
		0 & \text{otherwise},
	\end{cases} \\
	\frac{\partial \pof{\nabla \cdot X^\gamma}_i}{\partial \rho_\ell} &= \frac{1}{2}\sum_{\substack{k \\ \pof{v_i, v_k} \in E_M}}\pof{\pof{\frac{\partial p_{i, k}}{\partial \rho_\ell} - \frac{\partial p_{c\pof{i, k}, i}}{\partial \rho_\ell}} \cdot X^\gamma_{i, k} + \pof{p_{i, k} - p_{c\pof{i, k}, i}} \cdot \frac{\partial X^\gamma_{i, k}}{\partial \rho_\ell}}, \\
	\frac{\partial \widetilde{\phi}^\gamma}{\partial \rho_\ell} &= \pof{L_C^{\text{N}}}^+\pof{\frac{\partial \pof{\nabla \cdot X^\gamma}}{\partial \rho_\ell} - \frac{\partial L_C^{\text{N}}}{\partial \rho_\ell}\phi^\gamma}, \\
	\frac{\partial \phi^\gamma}{\partial \rho_\ell} &= \frac{\partial \widetilde{\phi}^\gamma}{\partial \rho_\ell} - \pof{\frac{\partial \widetilde{\phi}^\gamma}{\partial \rho_\ell}}_{\gamma}.
\end{align*} Note that \(\gamma = \mathrm{arg\,min}\pof{\phi}\), which is where the final subtraction comes from.


	\subsection{Geodesic Loss}
	We will define the following in this section: \begin{center}\begin{tabular}{r|l}
	\(\widetilde{\phi}\) & Geodesic distances corresponding to edges in \(E_G\) \\ \hline
	\(\widetilde{d}\) & Centered version of \(\widetilde{\phi}\) \\ \hline
	\(d\) & Normalized and centered version of \(\widetilde{\phi}\) \\ \hline
	\(\beta\) & The least squares linear estimator between \(\widetilde{\phi}\) and \(t\) \\ \hline
	\(L_{\mathrm{geodesic}}\) & The sum of squared residuals when using \(\beta\) as an estimator
\end{tabular}\end{center} In this section, we will abuse notation a bit and write things like \(\phi_e\) to mean \(\phi_{i, j}\), where \(e = \pof{i, j} \in E_G\).

\subsubsection{Forward Computation}
We make the following computations: \begin{align*}
	\widetilde{\phi}_e &= \phi_e \text{ when \(e \in E_G\)}, \\
	\widetilde{d} &= \widetilde{\phi} - \frac{1}{\abs{E_G}}\pof{\widetilde{\phi} \cdot \mathbf{1}}\mathbf{1},
\end{align*}

\subsubsection{Reverse Computation}
\end{document}
