Some mesh notation first. If \(i\) and \(j\) are two indices vertices for which \(\pof{v_i, v_j}\) is an edge, let \(c\pof{i, j}\) be the index such that \(v_i \to v_j \to v_{c\pof{i, j}}\) traces a triangle counterclockwise. Note that this index exists and is unique assuming we have a mesh without boundary.

We define the following variables:
\begin{center}\begin{tabular}{r|l}
	\(N_{i, j}\) & Outward normal of triangle \(v_i \to v_j \to v_{c\pof{i, j}}\) \\ \hline
	\(A_{i, j}\) & Area of triangle \(v_i \to v_j \to v_{c\pof{i, j}}\) \\ \hline
	\(D_{i, j}\) & Vertex triangle areas; diagonal \\ \hline
	\(\theta_{i, j}\) & Measure of \(\angle v_iv_{c\pof{i, j}}v_j\) \\ \hline
	\(L_C\) & Cotangent operator; sparse \\ \hline
	\(L_C^{\text{Neumann}}\) & Cotangent operator with \href{https://en.wikipedia.org/wiki/Neumann_boundary_condition}{zero-Neumann boundary condition} \\ \hline
	\(L_C^{\text{Dirichlet}}\) & Cotangent operator with \href{https://en.wikipedia.org/wiki/Dirichlet_boundary_condition}{zero-Dirichlet boundary condition} \\ \hline
\end{tabular}\end{center}

\subsubsection{Forward Computation}

We have the following (standard) definition of the Laplace-Beltrami operator on a mesh:

\begin{align*}
	N_{i, j} &= \pof{v_i - v_{c\pof{i, j}}} \times \pof{v_j - v_{c\pof{i, j}}} \\
	A_{i, j} &= \frac{1}{2}\|N_{i, j}\|_2 \\
	D_{i, j} &= \begin{cases}
		\displaystyle\frac{1}{3}\sum_{\substack{k \\ \text{\(\pof{v_i, v_k}\) is an edge}}}A_{i, k} & \text{if \(i = j\)}, \\
		0 & \text{otherwise},
	\end{cases} \\
	\cot\pof{\theta_{i, j}} &= \frac{\pof{v_i - v_{c\pof{i, j}}} \cdot \pof{v_j - v_{c\pof{i, j}}}}{2A_{i, j}} \\
	\pof{L_C^{\text{Neumann}}}_{i, j} &= \begin{cases}
		\displaystyle\frac{1}{2}\pof{\cot\pof{\theta_{i, j}} + \cot\pof{\theta_{j, i}}} & \text{if \(\pof{v_i, v_j}\) or \(\pof{v_j, v_i}\) is a half-edge}, \\
		\displaystyle-\frac{1}{2}\sum_{\substack{k \\ \text{\(\pof{v_i, v_k}\) or \(\pof{v_k, v_i}\)} \\ \text{is a half-edge}}}\pof{\cot\pof{\theta_{i, k}} + \cot\pof{\theta_{k, i}}} & \text{if \(i = j\)}, \\
		0 & \text{otherwise},
	\end{cases} \\
	\pof{L_C^{\text{Dirichlet}}}_{i, j} &= \begin{cases}
		\displaystyle\frac{1}{2}\pof{\cot\pof{\theta_{i, j}} + \cot\pof{\theta_{j, i}}} & \text{if \(\pof{v_i, v_j}\) and \(\pof{v_j, v_i}\) are half-edges}, \\
		\displaystyle-\frac{1}{2}\sum_{\substack{k \\ \text{\(\pof{v_i, v_k}\) and \(\pof{v_k, v_i}\)} \\ \text{are half-edges}}}\pof{\cot\pof{\theta_{i, k}} + \cot\pof{\theta_{k, i}}} & \text{if \(i = j\)}, \\
		0 & \text{otherwise}.
	\end{cases}
\end{align*}

Flipping our attention back to meshes without boundary, the two definitions above coincide, so we can write \[L_C = L_C^{\text{Neumann}} = L_C^{\text{Dirichlet}}.\]

\subsubsection{Reverse Computation}

For the ease of notation, assume that we are using the spherical setup, so \(v_\ell = \rho_\ell s_\ell\).

We compute

\begin{align*}
	\frac{\partial v_i}{\partial \rho_\ell} &= \begin{cases}
		s_i & \text{if \(\ell = i\)}, \\
		0 & \text{otherwise},
	\end{cases} \\
	\frac{\partial N_{i, j}}{\partial \rho_\ell} &= \begin{cases}
		\pof{v_{c\pof{i, j}} - v_j} \times \frac{\partial v_\ell}{\partial \rho_\ell} & \text{if \(\ell = i\)}, \\
		\pof{v_i - v_{c\pof{i, j}}} \times \frac{\partial v_\ell}{\partial \rho_\ell} & \text{if \(\ell = j\)}, \\
		\pof{v_j - v_i} \times \frac{\partial v_\ell}{\partial \rho_\ell} & \text{if \(\ell = c\pof{i, j}\)}, \\
		0 & \text{otherwise},
	\end{cases} \\
	\frac{\partial A_{i, j}}{\partial \rho_\ell} &= \frac{1}{4A_{i, j}}N_{i, j} \cdot \frac{\partial N_{i, j}}{\partial \rho_\ell}, \\
	\pof{\frac{\partial D}{\partial \rho_\ell}}_{i, j} &= \begin{cases}
		\displaystyle\frac{1}{3}\sum_{\substack{k \\ \text{\(\pof{v_i, v_k}\) is an edge}}}\frac{\partial A_{i, k}}{\partial \rho_\ell} & \text{if \(i = j\)}, \\
		0 & \text{otherwise},
	\end{cases} \\
	\frac{\partial}{\partial \rho_\ell}\cot\pof{\theta_{i, j}} &= \begin{cases}
		\displaystyle\frac{\pof{v_j - v_{c\pof{i, j}}} \cdot \frac{\partial v_\ell}{\partial \rho_\ell} - 2\cot\pof{\theta_{i, j}}\frac{\partial A_{i, j}}{\partial \rho_\ell}}{2A_{i, j}} & \text{if \(\ell = i\)}, \\
		\displaystyle\frac{\pof{v_i - v_{c\pof{i, j}}} \cdot \frac{\partial v_\ell}{\partial \rho_\ell} - 2\cot\pof{\theta_{i, j}}\frac{\partial A_{i, j}}{\partial \rho_\ell}}{2A_{i, j}} & \text{if \(\ell = j\)}, \\
		\displaystyle\frac{\pof{2v_{c\pof{i, j}} - v_i - v_j} \cdot \frac{\partial v_\ell}{\partial \rho_\ell} - 2\cot\pof{\theta_{i, j}}\frac{\partial A_{i, j}}{\partial \rho_\ell}}{2A_{i, j}} & \text{if \(\ell = c\pof{i, j}\)}, \\
		0 & \text{otherwise},
	\end{cases} \\
	\pof{\frac{\partial L_C^{\text{Neumann}}}{\partial \rho_\ell}}_{i, j} &= \begin{cases}
		\displaystyle\frac{1}{2}\pof{\frac{\partial}{\partial \rho_\ell}\cot\pof{\theta_{i, j}} + \frac{\partial}{\partial \rho_\ell}\cot\pof{\theta_{j, i}}} & \text{if \(\pof{v_i, v_j}\) or \(\pof{v_j, v_i}\) is a half-edge}, \\
		\displaystyle-\frac{1}{2}\sum_{\substack{k \\ \text{\(\pof{v_i, v_k}\) or \(\pof{v_k, v_i}\)} \\ \text{is a half-edge}}}\pof{\frac{\partial}{\partial \rho_\ell}\cot\pof{\theta_{i, k}} + \frac{\partial}{\partial \rho_\ell}\cot\pof{\theta_{k, i}}} & \text{if \(i = j\)}, \\
		0 & \text{otherwise},
	\end{cases} \\
	\pof{\frac{\partial L_C^{\text{Neumann}}}{\partial \rho_\ell}}_{i, j} &= \begin{cases}
		\displaystyle\frac{1}{2}\pof{\frac{\partial}{\partial \rho_\ell}\cot\pof{\theta_{i, j}} + \frac{\partial}{\partial \rho_\ell}\cot\pof{\theta_{j, i}}} & \text{if \(\pof{v_i, v_j}\) and \(\pof{v_j, v_i}\) are half-edges}, \\
		\displaystyle-\frac{1}{2}\sum_{\substack{k \\ \text{\(\pof{v_i, v_k}\) and \(\pof{v_k, v_i}\)} \\ \text{are half-edges}}}\pof{\frac{\partial}{\partial \rho_\ell}\cot\pof{\theta_{i, k}} + \frac{\partial}{\partial \rho_\ell}\cot\pof{\theta_{k, i}}} & \text{if \(i = j\)}, \\
		0 & \text{otherwise}.
	\end{cases}
\end{align*}